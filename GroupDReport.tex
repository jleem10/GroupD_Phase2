\documentclass[a4paper, 12pt]{report}
\usepackage[top = 2.54cm, bottom = 2.54cm, left = 2.54cm, right = 2.54cm]{geometry}
\usepackage{graphicx}
\usepackage{enumerate}
\usepackage{amsmath}
\usepackage{verbatim}
\usepackage{fancyhdr}
\usepackage{fixltx2e} % Package for subscripting and superscripting... %
\pagestyle{fancy}
\setlength{\parindent}{0pt}
\lhead{}
\rhead{Group D - Phase 2 Report}
\renewcommand{\headrulewidth}{0.3pt}

\title{Influenza Virus Infection Modeling}
\author{A.~Ambuehl -- \texttt{antonietta.ambuehl@dtc.ox.ac.uk} \and J.~Leem -- \texttt{jinwoo.leem@dtc.ox.ac.uk} \and M.~Lucken -- \texttt{malte.lucken@dtc.ox.ac.uk} \and W.~Smith -- \texttt{william.smith@dtc.ox.ac.uk} \and O.~Thomas -- \texttt{owen.thomas@dtc.ox.ac.uk} \\\\
University of Oxford DTC \\
Rex Richards Building \\
South Parks Road\\
\underline{Oxford, OX1 3QZ, United Kingdom}\\
}

\begin{document}
\maketitle

%% STYLISTIC COMMENTS %%
%
% Please format function names in computer modern (>> "\textt{}") and latin (i.e., etc., via, e.g.) in italic (>> "\textit{}")
% Citations take >> "glim~\cite{paper}" to give proper spacing.
% Acronymns (as with any other object) should not be pluralised using an apostrophe. (so ODEs not ODE's)
% Don't use the regular speechmarks, they are not formatted correctly (so ``speech'  not "speech"')
% Strings of references take individual boxes (so \cite{X}~\cite{Y}~\cite{Z} rather than \cite{X,Y,Z})
% MatLab is refered to as ">> MatLab\textsuperscript{\textregistered} "

%%Add these to the Bib

%@article{Perelson,
  %title={Modelling viral and immune system dynamics},
  %author={Perelson, A.S. and others},
  %journal={Nature Reviews Immunology},
  %volume={2},
  %number={1},
  %pages={28--36},
  %year={2002}
%}


\chapter{Background and Aims} %Done by Jin
\section{Biological Problem and Previous Work}

The influenza virus is responsible for considerable public discomfort, and poses the risk of seasonal pandemic and a potentially major health hazard to the elderly or infirm.~\cite{Smith} For the purposes of treating and preventing viral infections, it would be desirable to generate a model which simulates the course of viral infection. Models that study the interplay between immune system components within an individual suffering from viral infection have previously been developed (for a review, see Perelson 2002~\cite{Perelson}), and have the potenital to capture multiple aspects of human epithelial immunology (\textit{e.g.} interaction of cytokines, effector cells, antibodies, virions, \textit{etc.}). 

In particular, a model was devised in 2007 which encompassed many of the factors involved in human-virus immune response ~\cite{Hancioglu}, such as:

\begin{itemize}
\item Antibodies and their affinity toward the circulating virus;
\item Antigen presentation;
\item Production and clearance of the virus, \textit{etc.}.
\end{itemize}

This paper of Hancioglu \textit{et al.}featured a continuum model based on a series of ordinary differential equations (ODEs). The model was used predict changes in the population of healthy and infected cells, and levels of the free virus over the course of infection. \\

In 2013, the results of the model have been reproduced by members of Group G by using a series of Matlab and C++ code ~\cite{GroupG}. This computational model successfully reproduces cellular and viral dynamics in a hypothetical infection scenario, displaying good fidelity. In addition, the authors provide an open-source implementation of the model in MatLab\textsuperscript{\textregistered}, featuring a graphic user interface (GUI) that enables users to control and systematically parse the model's 27 parameters, as well as facilitating the visualisation of multiple datasets. \\

\section{Idenitfication of areas for extension}

Although the model is designed so that parameters can be tailored to each patient's immune capacity and each virus strain's virulence, it only reflects the natural biological response (\textit{i.e.} the adaptive and innate immune responses) to the virus. Moreover, the model was entirely deterministic: real biological processes tend to involve considerable noise, and several of the processes included in the system are thought to be intrinsically stochastic. Finally, we noted that the mode ommited certain interactions between immune system components.
Our group felt that the model could be extended further by:
\begin{enumerate}
\item Considering the effects of anti-influenza drugs on the cellular and viral dynamics;
\item Converting the coupled ODE representation to include stochastic noise;
\item Including the influence of helper T-cells (T\textsubscript{H} cells), which help to form plasma cells.
\end{enumerate}
In addition, we scrutinised the code of their implementation closely, and investigated ways in which it could be rewritten in a more efficient manner. Ultimately, these extensions culminated to a new model as shown in Figure ~\ref{newmodel}:

\begin{figure}[htb]
\label{newmodel}
\includegraphics[width = 150mm]{Diagram.png}
\caption{Each of the relevant compartments, as per the original model, are labelled in their respective abbreviations (\emph{e.g.} Infected cells are represented by $I$ and antigen presenting cells (APCs) are represented by $M$, \emph{etc.}). Our additions to the model are labelled as orange compartments with dotted lines for their relevant processes. These new additions include: drugs (neuraminidase inhibitors, $D_1$, and amantadine $D_2$), helper T cell:B cell dynamics (T\textsubscript{FH} and B) and introduction of stochasticity (labelled with yellow stars).}
\end{figure}

\chapter{Extension of the Model}
\section{Extension of the Model I -- Modelling Antiviral therapy}

To extend the model, we decided to investigate the effects of adding an antiviral agent to the system.
Antivirals are drugs used to control viral infections both therapeutically and prophylactically. They operate by interfering with the virus copying sequence at one or more points in its replication cycle. For example, two common antiviral targets are:
\begin{enumerate}
\item \textbf{Viral M2 proton channels:} Disruption of viral unpackaging in host cytosol \textit{via} the competitive inhibition of the viral M2 proton channel;~\cite{Hu}
\item \textbf{Viral Neuraminidase:} Prevention of viral budding \textit{via} competitive inhibition of the neuraminidases responsible for severing newly-created virus particles from their host cells.\cite{Satoh}  
\end{enumerate}
Instances of antivirals exploiting the above mechanisms include  Amantadine (trade name ``Symmetrel'') and Oseltamivir phosphate (``Tamiflu'').\\
%Tamiflu works by inhibiting the action of the enzyme viral neuraminidase, whose action is necessary to allow newly-created viral particles to detach from their infected hosts.S
%Symmetrel, an older drug, operates by interfering with the action of viral M2 proton channels, preventing virus particles from becoming decoated once they are absorbed into cells by endocytosis.

\subsection{Including Drug Variables in Existing Model}

Kinetic models have previously been used to investigate the influence of an antiviral drug on viral dynamics within infected individuals -- such as the recent work of Smith \textit{et al.}~\cite{Smith} who modelled drug influences in simple kinetic models.
The influence of the 2 drug types was accommodated by changing terms in equations (1), (2) and (3), to redescribe the relationship between the variables $H$ (healthy cells), $I$ (Infected cells) and $V$ (Virus particles):

\begin{itemize}
\item  $\gamma_{HV}VH \rightarrow \gamma_{HV}(1 - \epsilon)VH$ in the case of Amantadinoid drugs;
\item $\gamma_{V}I \rightarrow \gamma_{V}(1 - \epsilon)I $ in the case of Neuroamidinase Inhibitors.
\end{itemize}

%$\epsilon\left(t\right)$
where $\epsilon$ represents the drug's efficacy,

\begin{equation}
\epsilon(t) = \frac{\epsilon_{max}D(t)^{n}}{D(t)^{n}+\mathrm{IC}^{n}_{50}}
\end{equation}

as a function of time.
In their 2011 paper, the authors assumed a drug concentration that was constant with respect to time. Adopting this simplification would have prevented us from analyzing effects such as the dependency of drug impact on the drug administration time.
We sought to extend this work by combining a more realistic, temporal drug model with the extended dynamic model of group G.\cite{Group G}

\begin{figure}[htb]
\begin{centering}
\includegraphics[width=150mm]{DrugProfile.png}
\caption{Plot of $\epsilon(t)$ against $t$ according to equation~\eqref{eq:drugeff}, for $c_1 = c_2 = c_3 = 1$.}
\label{fig:DrugProfile}
\end{centering}
\end{figure}

To achieve this, we used a time-dependent expression for $\epsilon$ of the form
\begin{equation}
\epsilon(t) = \frac{c_{1}t}{1 + c_{2}t+c_{3}^{2}t^{2}}. \label{eq:drugeff}
\end{equation}
Equation \eqref{eq:drugeff} seeks to emulate a realistic drug concentration profile in infected tissue.\footnote{This equation has a similar form to analytical solutions to a 2-compartment Pharmacokinetic model, such as the one reported in \cite{}}
A rapid initial surge in drug concentration following administration ($t = 0$), is followed by a slow, exponential purging of the drug from the tissues as it is excreted (figure~\ref{fig:DrugProfile}). The parameters in equation~\ref{eq:drugeff} can be found by fitting to known experimental data pertaining to the drug's pharmacokinetic profile.
% by Antonietta
Specifically, its peak time $t_{peak}$, when the full concentration $d_{max}$ of the drug has been released, and its half life $t_{half}$, which is the time after administration at which the concentration of the drug has reached half its maximal value. The constants $c_1$, $c_2$ and $c_3$ relate to the experimental data $d_{max}$, $t_{peak}$ and $t_{half}$ as follows:
\begin{eqnarray}
c_3 &=& \frac{1}{t_{peak}}\\
c_2 &=& \frac{(c_3 t_{half})^2 - 4 c_3 t_{half} + 1}{t_{half}}\\
c_1 &=& d_{max} (2 c_3 + c_2)
\end{eqnarray}
Also, in order to control the time of administration, a further parameter $t_{on}$ has to be introduced, and equation \eqref{eq:drugeff} changed to the form
\begin{equation}
\epsilon(n) = \left\{ 
   \begin{array}{l l}
     0 & \quad \text{for} \quad 0 \leq t \leq t_{on}\\
     \frac{\displaystyle c_{1}(t-t_{on})}{\displaystyle 1 + c_{2}(t-t_{on})+c_{3}^{2}(t-t_{on})^{2}} & \quad \text{for} \quad t > t_{on}
   \end{array} \right. \label{eq:drugadmin}
\end{equation}
\begin{figure}[htb]
\begin{centering}
\includegraphics[width=150mm]{drugAdministration.png}
\caption{Plot of $\epsilon(t)$ against $t$ according to equation~\eqref{eq:drugadmin}. The input parameters for this graph are: $d_{max} = 2$, $t_{on} = 3$, $t_{peak} = 0.5$, $t_{half} = 0.9$, which results in $c_1 = 1.42$, $c_2 = -3.29$, $c_3 = 2$.}
\label{fig:DrugProfile}
\end{centering}
\end{figure}

Key questions include:
\begin{enumerate}
\item How does the time delay between infection onset and drug administration effect the treatment outcome?
\item Can polytherapy exhibit synergy? \textit{I.e.} could 2 drugs working together ever achieve more than the sum of their individual effects?
\end{enumerate}

\subsubsection{Modelling Viral Drug resistance}

Viruses such as Influenza A are known to be capable of rapidly developing resistance both to Tamilfu and Symmetrel.
Indeed, innate viral resistance to Symmetrel is now so widespread that it has been withdrawn as a drug.
We sought to model the onset of viral drug resistance by splitting the viral populaton $V$ into subpopulations of different mutants, displaying increasing levels of drug resistance\footnote{The drug resistance was modelled as having an associated fitness cost, making mutation favourable in the presence of antivirals but marginally unfavourable in their absence.}. Initially, all viruses were of an unmutated type, but subsequent mutations (modelled by stochastic population transfers) allow occasional transfer to progressively more resistant types which then proliferate faster owing to the selective pressure imposed by the drug.
%% Are we even including this?

\section{Consideration of T\textsubscript{H} cells in Plasma Cell Dynamics} %By Jin
In the original model, the interaction of T\textsubscript{H}1 and T\textsubscript{H}2 cells were omitted with almost no justification - this seems unusual, especially considering how critical T\textsubscript{H} cells are fundamental to the clonal expansion of B-cells and their transition into plasma cells. Interestingly, new research suggests that it's neither T\textsubscript{H}1 or T\textsubscript{H}2 cells; instead, another cell type, the follicular helper T-cell, T\textsubscript{FH}, facilitates the switch ~\cite{Swain}. Though it would be ideal to integrate T\textsubscript{FH}:B-cell interaction into the model to enhance its biological relevance, their interaction has not yet been substantially modelled and our group felt it was precocious to propose an entirely new equation to model this interaction. On the other hand, the interaction of T-cells and B-cells has been modelled elsewhere ~\cite{Carneiro1}~\cite{Carneiro2}.\\ 

The change in T-cell population to antigen $k$ (\emph{i.e.} $\frac{dT_k}{dt}$) is dependent on the T-cell death rate ($k_{DT}$), proliferation rate ($k_{PT}$), thymic output rate ($k_{ST}$) and the function $\alpha{T}$ which depends on $\pi_{k}$ (stimulatory signal for T-cell activation), $n_{k}$ (inhibitory signal for T-cell activation), $T_{k}$, and two constants $a_{T1}$ and $a_{T2}$, hence leading to the equation:

\begin{eqnarray}
\frac{dT_k}{dt} &=& -k_{DT} \cdot T_k + k_{PT} \cdot \alpha_{T}(\pi_{k}, n_{k}, T_{k}) + k_{ST}; \label{dTK} \\
\alpha_{T}(\pi_{k}, n_{k}, T_{k}) &=& exp\left[-\left(\frac{log(n_{k}) - a_{T1}}{a_{T2}}\right)^{2}\right]
\end{eqnarray}

The population of active B cells, $B_{i}$, is dependent on the T-cell population; like equation ~\ref{dTK}, $B_{i}$ depends on B-cell death rate ($k_{DB}$), proliferation rate ($k_{PB}$), and the function $\alpha{B}$ (which depends on $\sigma_{i}$ [B cell induction signal], $\tau_{i}$ [number of active T cells], $\beta_{i}$ [the number of induced B cells] and $B_{i}$), thus leading to:
\begin{eqnarray}
\frac{dB_i}{dt} &=& -k_{DB} \cdot B_i + k_{PB} \cdot \alpha_{B}(\sigma_{i}, \tau_{i}, B_{i}); ~\label{dBI} \\
\alpha_{B}(\sigma_{i}, \tau_{i}, B_{i}) &=& \frac{\tau_{i}\cdot{\beta_{i}}}{\tau_{i} + \beta_{i}}, \\
\beta_{i} &=&  exp\left[-\left(\frac{log(\sigma_{i}) - b_{1}}{b_{2}}\right)^{2}\right] \cdot B_{i}
\end{eqnarray}

Surprisingly, for the purposes of analysis, the authors simplify both ~\ref{dTK} and ~\ref{dBI} assuming that there is at least one activated T-cell, thus reducing this system to:
\begin{eqnarray}
\frac{dT_k}{dt} &=& (k_{PT}-k_{DT}) \cdot T_k \\ ~\label{dTKSimple}
\frac{dB_i}{dt} &=& -k_{DB} \cdot B_i + k_{PB} \cdot \frac{T_{k} \cdot B_{i}}{T_{k}+B_{i}}; ~\label{dBISimple}
\end{eqnarray}

Given this new simplified formula, we decided to assimilate these two equations into the existing computational model. Unfortunately, one of the biggest challenges was actually fitting the newly-defined T\textsubscript{FH} cells into the overall scheme in Figure ~\ref{newmodel}. Considering that the mechanism of T\textsubscript{FH} cell differentiation remains undefined and the possibility that T\textsubscript{FH} cells are activated by B cells ~\cite{Crotty}, the equations that we are using may be inaccurate. On the other hand, there is sufficient evidence to suggest that T\textsubscript{FH} cells are activated by signals derived from APCs - thus, continuing with this notion, we suggest that equation ~\ref{dTKSimple} is scaled by the population of APCs, $M$.

\begin{equation}
\frac{dT_k}{dt} = (k_{PT}-k_{DT}) \cdot T_k \cdot M \\
\end{equation}

Under the new scheme, na\"\i ve B-cell activation by T\textsubscript{FH} cells can be represented by equation ~\ref{dBISimple}. As for the subsequent formation of plasma cells from B cells, we can see the original ODE for the formation of plasma cells (where $b_{PM}$ and $\alpha{P}$ are constants representing plasma cell synthesis and death, respectively):
\begin{equation}
\frac{dP}{dt} = b_{PM}MP + \alpha_{P}(1 - P)
\end{equation}

We decided to modify this equation slightly; we retained the constants as they represent properties intrinsic to the \emph{plasma cell}. However, we changed the dependence of plasma cell synthesis to the population of active B cells, $B_{i}$, rather than M (\emph{N.B. }To avoid confusion in nomenclature, $b_{PM}$ was renamed to $b_{PB}$). Therefore, we would attain:
\begin{equation}
\frac{dP}{dt} = b_{PB}B_{i}P + \alpha_{P}(1 - P)
\end{equation}

In summary, to account for the role of helper T cells in plasma cell synthesis with the influenza virus (V) as the antigen, we propose the following system of ODEs:
\begin{eqnarray*}
\frac{dT_V}{dt} &=& (k_{PT}-k_{DT}) \cdot T_V \cdot M \\
\frac{dB_i}{dt} &=& -k_{DB} \cdot B_i + k_{PB} \cdot \frac{T_{V} \cdot B_{i}}{T_{V}+B_{i}} \\
\frac{dP}{dt} &=& b_{PB}B_{i}P + \alpha_{P}(1 - P)
\end{eqnarray*}

\section{Introduction of Stochasticity} %Done by Jin and Owen
To improve the current computational model, we felt that giving the user the freedom to incorporate stochasticity into their analyses was essential. The idea was not to replace the code \emph{per se}, but instead give the user a more realistic account of how cell populations and viral titre can vary over time. We decided that the two areas in which introduction of stochasticity were most biologiclaly justified were in virus production and development of antibody affinity. It is understood that there is a wide range of variability in virus production from infected cells \cite{MitchellJanuary152011}, so we felt justified in adding a noise term which affects the total virus population V. Similarly, the model already incorporates the affinity maturation of antibodies, but fails to model the fundamentally stochastic nature of the recombination process. In theory, every antibody is constructed from a diverse genetic framework and every antibody has a different affinity for the virus - only the antibody that binds strongest is selected for further expansion in the geminal centres \cite{Grimaldi15022005}. Consequently, we felt that another target for noise terms was the rate of change of variable S, in order to incorporate the iterative, random process of antibody selection.

There were two possible ways of approaching this problem; it is possible to introduce stochastic noise into either the cell populations or the dynamical rate paramters. Both were explored, and are discussed below.

\subsection{Stochastic Populations}

The study of stochastic differential equations (SDEs) is a widely studied topics in mathematics, with many resources for different algorithms and implementations. The ideal is to solve a differential equation with a well-defined noise function adding a stochastic element over time:

\begin{align}
\frac{\mathrm d X}{\mathrm d t}(t) = \mathrm f (X) + \text{``noise''}
\end{align}

In the above, $f(X,t)$ is the ``drift function" of the system that would comprise a traditional ODE and $g(t)$ is the component that contributes the stochastic noise with time. It is possible to introduce a traditional finite-difference forward-regression approach, by implementing the Euler-Maruyama method for an SDE, such that:

\begin{align}
X(t+\delta t) = X(t) + (\delta t)\, \mathrm f (X(t)) + \sqrt{ \delta t}\, \mathrm G (X(t))\, \eta \quad \text{such that} \quad \eta \in N(0,1)
\end{align}

Here, $N(0,1)$ is a normal distribution, with a mean of zero and standard deviation of one and $G(X(t))$ is the ``diffusion coefficient'' for our system. Given access to the functions $f(x)$ and $G(X)$ and a normally distributed random number generator, it is possible to solve this iteratively, with each new timestep explicitly using the previous one. This is a simple but reliable implementation, which represents a good method for introducing background noise into the concentrations of various variables in our system.

\subsection{Stochastic Rates}

The next approach, which is possibly more theoretically justifiable, is to allow the parameters of the ODE system to vary stochastically with each call of the function $f(X(T))$. This allows the code to isolate the mechanisms are contributing noise, rather than simply adding some arbitrary amount of stochasticity to the variable each timestep. This method allows for the standard library of ODE solving regression algorithms, since it is the drift function that contributes noise rather than the finite different method itself. It is consequently slightly easier to implement in various languages.

\subsection{Implementation in Matlab}

These two approaches were implemented in two different solver functions: \texttt{sdesolver()} and \texttt{sderatesolver()}, respectively. These resembled the native Matlab solvers, but with an extra input, describing the nature of the diffusion coefficient $G(X(t))$. \texttt{sdesolver()} is handed a two by ten dimensoinal matrix containing the vectors $a$ and $b$, used such that $G(X)=a.$*$X.^b$, elementwise. This gives a wide sampling of potential diffusion functions, allowing for easy sampling of interesting parameter spaces. The second function, \texttt{sderatesolver()}, demands two inputs, which specify the strength of the noise added to the creation of V and S, these being judged to be the processes most likely to realistically exhibit highly stochastic behaviour. A difficulty of implementation involved forcing all variables to remain non--negative: if either the numerical methods or stochastic noise push any variables below zero into an unphysical region of phase space, the variables would quickly and collectively diverge into positives or negative infinities. This was avoided with a simple check and modification.

\section{Results}

We observed several interesting phenomena over the course of exploring parameter and solver configurations for the SDEs. These can be loosely grouped into two classes, thodse of noise propagation and virus resurgence.

\subsection{Noise transferrance}

An interesting question which we addressed is the extent to which noise introduced to one parameter will effect the output of another parameter. Choosing S and V as being test cases, we added noise to each independantly and both together, observing the extent to which this effect the behaviour of other variables. We found that noise added to V became manifest in slight perturbations to the response curves of H, I and M., although the curves were still qualitatively the same objects. Noise added to the variable S had no obvious effect: desite being relatively well--connected in the system of ODEs, the other variables were less effected by perturbations in S. This may be owing to the role of S as a logical variable, ranging from its steady states at zero and one; stochastic variation these may be insignificant compared to the overall logical state of the efficacy.

\subsection{Virus Resurgence}

One dynamically significant effect of adding noise to either the virus population V or the rate of creation of V by infected cells is the occurance of virus resurgence and infection. If the noise introduced is significanly large then the virus population plotted against time will feature more than one spike, representing a reappearance of the virus population in the cell population. I speculate that this is the stochasticity pushing the depleted virus count above a threshold, sparking another wave of infection when the antibody count has decayed sufficiently. This new wave is accompanied by a corresponding rise in antibodies in response, hoping to combat the infection. It is interesting that the subsequent virus and antibody peaks are significantly lower than the first wave's: this can be explained by reference to the antibody efficacy S. This grows every time a new infection appears, describing the cell population's capacity to ``remember'' how to combat the virus, making the virus less successful and demanding less antibody concetration for each wave. These effects can be observed in Figure \ref{resurgance}.

\begin{figure}
\begin{center}
\includegraphics[width=120mm]{Resurgance_scissored.pdf}
\label{resurgance}
\caption{Outputs for V, A and S against time, using the default parameter set and noise on V scaling linearly with V (NoiseParameters(1,1)=NoiseParameters(1,2))}
\end{center}
\end{figure}


% -Relevance of Stochastic Modeling of the System
% -Explain the use of SDE's
% -Refinement of Code architecture

\chapter{Our code in comparison to Group G}

\bibliography{GroupDReportBibTex}
\bibliographystyle{plain}

\end{document}